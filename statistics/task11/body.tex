\begin{problem}
	При испытании нового лекарства от шонибудилеза пациентов разбили на
	две группы по 50 человек. Одной группе давали новое лекарство, а другой — крашеный
	сахар (при этом все пациенты содержались вместе, остальные процедуры проводились
	одинаково и ни сами пациенты, ни сестры не знали, кому что дают). В результате в
	экспериментальной группе выздоровело 42 человека, а в контрольной 35. Проверьте с
	5\%-м уровнем значимости гипотезу о том, что новое лекарство эффективнее плацебо.
	Вычислите p-value.
\end{problem}
	В качестве гипотезы $H_0$ возьмём гипотезу о том что вероятность выздороветь одинаковая, т.е. $p_1 = p_2$, в качестве альтернативной гипотезы $H_1$ то, что $p_1>p_2$
	
	\[ p_1 = \frac{42}{50} = 0.84\]
	\[ p_2 = \frac{35}{50} = 0.7\]
	\[ p = \frac{50 \cdot 42 + 50\cdot 35}{100 } = 0.77 \]
	\[ z = \sqrt{50\cdot50/100}\cdot \dfrac{p1-p2}{\sqrt{p\cdot(1-p)}} = 1.66337
	 \]
	 Статистика z имеет нормальное распределение 
	 Возьмём в качестве критической области $S = (z_{1-\alpha}, \infty) = (1.644854, \infty)$, тогда гипотеза $H_0$ отвергается в пользу гипотезы $H_1$.
	 
	 Значение $p-value = 1 - F(z) = 0.04811924$.
	 
\begin{problem}
	В двух параллельных классах 25 и 28 учеников соответственно. На медосмотре всем измерили рост. получилось, что в первом классе средний рост составил
	152 см со стандартным отклонением 4 см, а во втором 148 см со стандартным отклонением 5 см. Считая распределение роста в обоих классах нормальным, проверить гипотезу
	о совпадении роста с 5\%-м уровнем значимости. Вычислить p-value.
\end{problem}
		В качестве гипотезы $H_0$ возьмём гипотезу о том что Математическое ожидание среднего роста одинаковое, т.е. $Ex_1 = Ex_2$, в качестве альтернативной гипотезы $H_1$ то, что $Ex_1 \ne Ex_2$, тогда 
		\[ N_1 = 25, N_2 = 28 \]
		
		\[ \overline{X_1} = 152,\overline{X_2} = 148 \]
		
		 \[ s_1 = 4, s_2 = 5 \]
		 \[ z =  \dfrac{\overline{X_2} - \overline{X_2}}{\sqrt{\dfrac{s_1^2}{N_1}+\dfrac{s_2^2}{N_2}}} = 3.511076\]
		 Статистика имеет распределение стьюдента со степенью свободы $\mu = 58.56253$.
		 Критическая область имеет вид $S = (-\infty, -2.001308) \cup(2.001308, \infty)$, поэтому гипотезу $H_0$ отвергаем в пользу альтернативной $H_1$.
		 Значение $p-value = 2 \cdot \min(0.999567,1 - 0.999567) = 0.000866$.
		 
		 
	 