\documentclass{amsart}
\usepackage{ifxetex}
\ifxetex
  \usepackage{fontspec}
  \usepackage{xunicode}
  \usepackage{xltxtra}
  \usepackage{xecyr}
  \setmainfont[Mapping=tex-text,Ligatures=TeX]{Droid Serif}
  \usepackage{polyglossia}
  \setdefaultlanguage{russian}
\else
  \usepackage[utf8]{inputenc}
  \usepackage[T2A]{fontenc}
  \usepackage[english,russian]{babel}
  \usepackage{concrete}
\fi
\usepackage{tikz}
\usepackage{amsthm,amsmath,amsfonts,amssymb}
\usepackage{fullpage}
\usepackage{eufrak}
\usepackage{listings}
\usepackage{color}
\usepackage{xcolor}

\newtheorem{problem}{Задача}

\begin{document}

  \definecolor{dkgreen}{rgb}{0,0.6,0}
  \definecolor{gray}{rgb}{0.5,0.5,0.5}
  \definecolor{mauve}{rgb}{0.58,0,0.82}  

  \newcommand{\problemset}[1]{
    
    \begin{center}
      \Large #1
    \end{center}
  }

  \lstset{ %
    language=C++,                % the language of the code
    basicstyle=\footnotesize,           % the size of the fonts that are used for the code
    numbers=left,                   % where to put the line-numbers
    numberstyle=\tiny\color{gray},  % the style that is used for the line-numbers
    stepnumber=1,                   % the step between two line-numbers. If it's 1, each line 
                                    % will be numbered
    numbersep=5pt,                  % how far the line-numbers are from the code
    backgroundcolor=\color{white},      % choose the background color. You must add \usepackage{color}
    showspaces=false,               % show spaces adding particular underscores
    showstringspaces=false,         % underline spaces within strings
    showtabs=true,                 % show tabs within strings adding particular underscores
    frame=single,                   % adds a frame around the code
    rulecolor=\color{black!10},        % if not set, the frame-color may be changed on line-breaks within not-black text (e.g. comments (green here))
    tabsize=2,                      % sets default tabsize to 2 spaces
    captionpos=b,                   % sets the caption-position to bottom
    breaklines=true,                % sets automatic line breaking
    breakatwhitespace=false,        % sets if automatic breaks should only happen at whitespace
    title=\lstname,                   % show the filename of files included with \lstinputlisting;
                                    % also try caption instead of title
    keywordstyle=\color{blue},          % keyword style
    commentstyle=\color{dkgreen},       % comment style
    stringstyle=\color{mauve},        % string literal style
    escapeinside={\%*}{*)},            % if you want to add LaTeX within your code
    morekeywords={done, to},              % if you want to add more keywords to the set
  %  deletekeywords={...}              % if you want to delete keywords from the given language
  }

  \begin{tabbing}
\hspace{11cm} \= Студент: \= Веслогузова Александра \\
  \> Группа: \> SE \\
  \> Дата: \> \today
\end{tabbing}
\hrule
\vspace{1cm}


  \begin{problem}
	При испытании нового лекарства от шонибудилеза пациентов разбили на
	две группы по 50 человек. Одной группе давали новое лекарство, а другой — крашеный
	сахар (при этом все пациенты содержались вместе, остальные процедуры проводились
	одинаково и ни сами пациенты, ни сестры не знали, кому что дают). В результате в
	экспериментальной группе выздоровело 42 человека, а в контрольной 35. Проверьте с
	5\%-м уровнем значимости гипотезу о том, что новое лекарство эффективнее плацебо.
	Вычислите p-value.
\end{problem}
	В качестве гипотезы $H_0$ возьмём гипотезу о том что вероятность выздороветь одинаковая, т.е. $p_1 = p_2$, в качестве альтернативной гипотезы $H_1$ то, что $p_1>p_2$
	
	\[ p_1 = \frac{42}{50} = 0.84\]
	\[ p_2 = \frac{35}{50} = 0.7\]
	\[ p = \frac{50 \cdot 42 + 50\cdot 35}{100 } = 0.77 \]
	\[ z = \sqrt{50\cdot50/100}\cdot \dfrac{p1-p2}{\sqrt{p\cdot(1-p)}} = 1.66337
	 \]
	 Статистика z имеет нормальное распределение 
	 Возьмём в качестве критической области $S = (z_{1-\alpha}, \infty) = (1.644854, \infty)$, тогда гипотеза $H_0$ отвергается в пользу гипотезы $H_1$.
	 
	 Значение $p-value = 1 - F(z) = 0.04811924$.
	 
\begin{problem}
	В двух параллельных классах 25 и 28 учеников соответственно. На медосмотре всем измерили рост. получилось, что в первом классе средний рост составил
	152 см со стандартным отклонением 4 см, а во втором 148 см со стандартным отклонением 5 см. Считая распределение роста в обоих классах нормальным, проверить гипотезу
	о совпадении роста с 5\%-м уровнем значимости. Вычислить p-value.
\end{problem}
		В качестве гипотезы $H_0$ возьмём гипотезу о том что Математическое ожидание среднего роста одинаковое, т.е. $Ex_1 = Ex_2$, в качестве альтернативной гипотезы $H_1$ то, что $Ex_1 \ne Ex_2$, тогда 
		\[ N_1 = 25, N_2 = 28 \]
		
		\[ \overline{X_1} = 152,\overline{X_2} = 148 \]
		
		 \[ s_1 = 4, s_2 = 5 \]
		 \[ z =  \dfrac{\overline{X_2} - \overline{X_2}}{\sqrt{\dfrac{s_1^2}{N_1}+\dfrac{s_2^2}{N_2}}} = 3.511076\]
		 Статистика имеет распределение стьюдента со степенью свободы $\mu = 58.56253$.
		 Критическая область имеет вид $S = (-\infty, -2.001308) \cup(2.001308, \infty)$, поэтому гипотезу $H_0$ отвергаем в пользу альтернативной $H_1$.
		 Значение $p-value = 2 \cdot \min(0.999567,1 - 0.999567) = 0.000866$.
		 
		 
	 

\end{document}
