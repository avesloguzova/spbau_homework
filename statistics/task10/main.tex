\documentclass{amsart}
\usepackage{ifxetex}
\ifxetex
  \usepackage{fontspec}
  \usepackage{xunicode}
  \usepackage{xltxtra}
  \usepackage{xecyr}
  \setmainfont[Mapping=tex-text,Ligatures=TeX]{CMU Serif}
  \usepackage{polyglossia}
  \setdefaultlanguage{russian}
\else
  \usepackage[utf8]{inputenc}
  \usepackage[T2A]{fontenc}
  \usepackage[english,russian]{babel}
  \usepackage{concrete}
\fi
\usepackage{tikz}
\usepackage{amsthm,amsmath,amsfonts,amssymb}
\usepackage{fullpage}
\usepackage{eufrak}
\usepackage{listings}
\usepackage{color}
\usepackage{xcolor}

\newtheorem{problem}{Задача}

\begin{document}

  \definecolor{dkgreen}{rgb}{0,0.6,0}
  \definecolor{gray}{rgb}{0.5,0.5,0.5}
  \definecolor{mauve}{rgb}{0.58,0,0.82}  

  \newcommand{\problemset}[1]{
    
    \begin{center}
      \Large #1
    \end{center}
  }

  \lstset{ %
    language=C++,                % the language of the code
    basicstyle=\footnotesize,           % the size of the fonts that are used for the code
    numbers=left,                   % where to put the line-numbers
    numberstyle=\tiny\color{gray},  % the style that is used for the line-numbers
    stepnumber=1,                   % the step between two line-numbers. If it's 1, each line 
                                    % will be numbered
    numbersep=5pt,                  % how far the line-numbers are from the code
    backgroundcolor=\color{white},      % choose the background color. You must add \usepackage{color}
    showspaces=false,               % show spaces adding particular underscores
    showstringspaces=false,         % underline spaces within strings
    showtabs=true,                 % show tabs within strings adding particular underscores
    frame=single,                   % adds a frame around the code
    rulecolor=\color{black!10},        % if not set, the frame-color may be changed on line-breaks within not-black text (e.g. comments (green here))
    tabsize=2,                      % sets default tabsize to 2 spaces
    captionpos=b,                   % sets the caption-position to bottom
    breaklines=true,                % sets automatic line breaking
    breakatwhitespace=false,        % sets if automatic breaks should only happen at whitespace
    title=\lstname,                   % show the filename of files included with \lstinputlisting;
                                    % also try caption instead of title
    keywordstyle=\color{blue},          % keyword style
    commentstyle=\color{dkgreen},       % comment style
    stringstyle=\color{mauve},        % string literal style
    escapeinside={\%*}{*)},            % if you want to add LaTeX within your code
    morekeywords={done, to},              % if you want to add more keywords to the set
  %  deletekeywords={...}              % if you want to delete keywords from the given language
  }

  \begin{tabbing}
\hspace{11cm} \= Студент: \= Веслогузова Александра \\
  \> Группа: \> SE \\
  \> Дата: \> \today
\end{tabbing}
\hrule
\vspace{1cm}


  \begin{problem}
	 Алекс заявляет, распределение баллов его студентов имеет дисперсию 100,
	 в то время как Тони уверен, что дисперсия гораздо больше. Помогите им разрешить
	 спор — проверьте заявление Тони как статистическую гипотезу с 10\%-м уровнем значимости, считая распределение баллов нормальным, если выборочная дисперсия для
	 случайно выбранных 10 студентов оказалась 162.
\end{problem}
Необходимо проверить значение дисперсии при неизвестном среднем.
В качестве гипотезы $H_0$ выберем $Dx = 100$, в качестве $H_1$ - $Dx > 100$ 

\[ \alpha = 0.10 \]
\[ N = 10 \]
\[ s^2(x) = 162 \]
\[ t = (N-1)\cdot \dfrac{s^2(x)}{100} = 9 \cdot \dfrac{162}{100} = 14.58\]
для одностороннего критерия критическая область имеет вид:
\[ S = (\chi_{1-\alpha,N-1},\infty) = (14.68366,\infty)\]

 Значения статистики не попадают в критическую область, поэтому нет оснований отвергнуть гипотезу $H_0$
 
 \begin{problem}
	 Производитель лекарства “Фунорен” (от головной боли) в своем буклете
	 утверждал, что “всего одна таблетка остановит боль в среднем меньше, чем за полчаса”. При работе с тестовой группой из 100 человек обнаружилось, что среднее время
	 до наступления эффекта составило 28.6 минуты со стандартным отклонением 4.2. Проверьте (с 5\%-м уровнем значимости) заявление производителя.
 \end{problem}
 Необходимо проверить значение среднего при неизвестной дисперсии. Предположим, что распределение нормально или асимптотически нормальное
 В качестве гипотезы $H_0$ выберем $Ex = 30$, в качестве $H_1$ - $Ex > 30$ 
 При нормальной аппроксимации критическая область имеет вид:
 \[ S = (z_{0.95},\infty) = (1.64, \infty)\]
 \[ t = \sqrt{N}\cdot \dfrac{\overline{x} - 30}{4.2} = -3.33\]
 
 Значения статистики не попадают в критическую область, поэтому нет оснований отвергнуть гипотезу $H_0$

\end{document}
