\begin{problem}
	 Алекс заявляет, распределение баллов его студентов имеет дисперсию 100,
	 в то время как Тони уверен, что дисперсия гораздо больше. Помогите им разрешить
	 спор — проверьте заявление Тони как статистическую гипотезу с 10\%-м уровнем значимости, считая распределение баллов нормальным, если выборочная дисперсия для
	 случайно выбранных 10 студентов оказалась 162.
\end{problem}
Необходимо проверить значение дисперсии при неизвестном среднем.
В качестве гипотезы $H_0$ выберем $Dx = 100$, в качестве $H_1$ - $Dx > 100$ 

\[ \alpha = 0.10 \]
\[ N = 10 \]
\[ s^2(x) = 162 \]
\[ t = (N-1)\cdot \dfrac{s^2(x)}{100} = 9 \cdot \dfrac{162}{100} = 14.58\]
для одностороннего критерия критическая область имеет вид:
\[ S = (\chi_{1-\alpha,N-1},\infty) = (14.68366,\infty)\]

 Значения статистики не попадают в критическую область, поэтому нет оснований отвергнуть гипотезу $H_0$
 
 \begin{problem}
	 Производитель лекарства “Фунорен” (от головной боли) в своем буклете
	 утверждал, что “всего одна таблетка остановит боль в среднем меньше, чем за полчаса”. При работе с тестовой группой из 100 человек обнаружилось, что среднее время
	 до наступления эффекта составило 28.6 минуты со стандартным отклонением 4.2. Проверьте (с 5\%-м уровнем значимости) заявление производителя.
 \end{problem}
 Необходимо проверить значение среднего при неизвестной дисперсии. Предположим, что распределение нормально или асимптотически нормальное
 В качестве гипотезы $H_0$ выберем $Ex = 30$, в качестве $H_1$ - $Ex > 30$ 
 При нормальной аппроксимации критическая область имеет вид:
 \[ S = (z_{0.95},\infty) = (1.64, \infty)\]
 \[ t = \sqrt{N}\cdot \dfrac{\overline{x} - 30}{4.2} = -3.33\]
 
 Значения статистики не попадают в критическую область, поэтому нет оснований отвергнуть гипотезу $H_0$